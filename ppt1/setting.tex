\usepackage[UTF8]{ctex}
\usepackage{commath}%定义d
\usepackage{graphicx}
\graphicspath{{fig/}}
\usepackage{booktabs} % To thicken table lines
\usepackage{bm}
\usepackage{cleveref}
\usepackage{hyperref}
\usepackage{amsmath}
\usepackage{listings}
\usepackage{latexsym}
\usepackage{minted}
\usepackage{amsmath}
\newcommand\mi{\mathrm{i}}
\newcommand\me{\mathrm{e}}


%代码设置
\usepackage{listings}
\usepackage{fontspec} % 定制字体
\newfontfamily\menlo{SFMono-Regular}
\usepackage{xcolor} % 定制颜色
\definecolor{mygreen}{rgb}{0,0.6,0}
\definecolor{mygray}{rgb}{0.5,0.5,0.5}
\definecolor{mymauve}{rgb}{0.58,0,0.82}
\lstset{
    % numbers=left,
    % numberstyle=\footnotesize\menlo,
    basicstyle=\footnotesize\menlo,
    backgroundcolor=\color{white},      % choose the background color
    columns=fullflexible,
    tabsize=4,
    breaklines=true,               % automatic line breaking only at whitespace
    captionpos=b,                  % sets the caption-position to bottom
    % commentstyle=\color{mygreen},  % comment style
    escapeinside={\%*}{*)},        % if you want to add LaTeX within your code
    keywordstyle=\color{blue},     % keyword style
    stringstyle=\color{mymauve}\ttfamily,  % string literal style
    frame=single,
    rulesepcolor=\color{red!20!green!20!blue!20},
    % identifierstyle=\color{red},
    language=c++,
    xleftmargin=1em,xrightmargin=2em, aboveskip=1em,
    framexleftmargin=2em,
    % numbers=left
}

%脚注
\renewcommand\thefootnote{\fnsymbol{footnote}}


%%% Maketitle metadata
\newcommand{\horrule}[1]{\rule{\linewidth}{#1}} 	% Horizontal rule
\newcommand{\tabincell}[2]{\begin{tabular}{@{}#1@{}}#2\end{tabular}}

