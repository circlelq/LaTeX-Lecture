\documentclass[12pt]{article}
\usepackage{natbib}
\usepackage{url}
\usepackage{stmaryrd}
\usepackage{mathrsfs}
\usepackage{amsmath}
\usepackage{graphicx}
\usepackage{parskip}
\usepackage{fancyhdr}
% \usepackage{underscore} % 下划线设置
\usepackage{commath}%定义d
\usepackage[UTF8,scheme = plain,scheme = chinese]{ctex}
\usepackage{geometry}
\usepackage{bm}
\usepackage{autobreak}
\usepackage{siunitx}
\usepackage{float}
\usepackage{subfig}
\usepackage{titlesec}
\usepackage{caption}
\usepackage{paralist}
\usepackage{multirow}
\usepackage{booktabs} % To thicken table lines
\usepackage{diagbox}
\usepackage{authblk}
\usepackage{indentfirst}
\usepackage{amsthm}
\usepackage{fontspec}
\usepackage{color}
%\usepackage{txfonts} %设置字体为times new roman
\usepackage{lettrine}
\usepackage{nameref}
%\usepackage[nottoc]{tocbibind}
\usepackage{amssymb}%font
\usepackage{lipsum}%make test words
\usepackage{picinpar}%words around the picture
\usepackage[all]{xy}%draw arrow
\usepackage{asymptote}%draw picture
\usepackage[perpage]{footmisc}%脚注每页清零
\usepackage{esint}
\renewcommand{\proofname}{\indent \sf \bfseries{证明}}

\catcode`\。=\active
\catcode`\,=\active
\catcode`\;=\active
\catcode`\:=\active
\newcommand{。}{.}
\newcommand{,}{,}
\newcommand{;}{;}
\newcommand{:}{:}

\geometry{bottom=3cm,left=3cm,right=3cm,top=3cm}
% \footskip = 60pt

% \setmainfont{TimesNewRomanPSMT}
% \setsansfont{Helvetica-Light}
% \setCJKmainfont[ItalicFont=STKaitiSC-Regular,BoldFont=STSongti-SC-Black]{STSongti-SC-Regular}
% \setCJKsansfont[BoldFont=STHeitiSC-Medium]{STHeitiSC-Light}


%\setmainfont{Times New Roman}

\ctexset{today=old}%日期类型设置

% ======================================
% = Color de la Universidad de Sevilla =
% ======================================
\usepackage{tikz}
\definecolor{PKUred}{cmyk}{0,1,1,0.45}

%超链接设置
\usepackage[breaklinks,colorlinks,linkcolor=PKUred,citecolor=PKUred,pagebackref,urlcolor=PKUred]{hyperref}
\usepackage{cleveref}
\newcommand{\crefpairconjunction}{ 和 }

% 节样式设置
% \newcommand{\hsp}{\hspace{20pt}}
% \newcommand{\nhsp}{\hspace{-30pt}}
% \titleformat{\section}{\Large\bfseries}{%\arabic{section}
%   \hspace{-22pt}\textcolor{PKUred}{\vrule width 2pt}\hsp}{0pt}{}


\titleformat{\subsection}
{\normalfont\large\bfseries}{}{0em}{}

% footnote 设置
% \renewcommand*\footnoterule{%
%   \vspace*{-3pt}%
%   {\color{PKUred}\hrule width 2in height 0.4pt}%
%   \vspace*{2.6pt}%
% }


%% Color the bullets of the itemize environment and make the symbol of the third
%% level a diamond instead of an asterisk.
%h\renewcommand*\textbullet{\dag}
\renewcommand*\labelitemi{\color{PKUred}\textbullet}
\renewcommand*\labelitemii{\color{PKUred}--}
\renewcommand*\labelitemiii{\color{PKUred}$\diamond$}
\renewcommand*\labelitemiv{\color{PKUred}\textperiodcentered}



%%% Equation and float numbering
\numberwithin{equation}{section}		% Equationnumbering: section.eq#
\numberwithin{figure}{section}			% Figurenumbering: section.fig#
\numberwithin{table}{section}				% Tablenumbering: section.tab#


%代码设置
\usepackage{listings}
\usepackage{fontspec} % 定制字体
% \newfontfamily\menlo{SFMono-Regular}
\usepackage{xcolor} % 定制颜色
\definecolor{mygreen}{rgb}{0,0.6,0}
\definecolor{mygray}{rgb}{0.5,0.5,0.5}
\definecolor{mymauve}{rgb}{0.58,0,0.82}
\lstset{
    numbers=left,
    numberstyle=\footnotesize\ttfamily,
    basicstyle=\footnotesize\ttfamily,
    backgroundcolor=\color{white},      % choose the background color
    columns=fullflexible,
    tabsize=4,
    breaklines=true,               % automatic line breaking only at whitespace
    captionpos=b,                  % sets the caption-position to bottom
    commentstyle=\color{mygreen},  % comment style
    escapeinside={\%*}{*)},        % if you want to add LaTeX within your code
    keywordstyle=\color{blue},     % keyword style
    stringstyle=\color{mymauve}\ttfamily,  % string literal style
    frame=single,
    rulesepcolor=\color{red!20!green!20!blue!20},
    % identifierstyle=\color{red},
    language=c++,
    xleftmargin=4em,xrightmargin=2em, aboveskip=1em,
    framexleftmargin=2em,
    numbers=left
}

% 脚注
% \renewcommand\thefootnote{\fnsymbol{footnote}}

%定义常数i、e、积分符号d
\newcommand\mi{\mathrm{i}}
\newcommand\me{\mathrm{e}}

%%% Maketitle metadata
\newcommand{\horrule}[1]{\rule{\linewidth}{#1}} 	% Horizontal rule
\newcommand{\tabincell}[2]{\begin{tabular}{@{}#1@{}}#2\end{tabular}}



% 章节标号深度
% \setcounter{secnumdepth}{2}

% 首行缩近
\setlength{\parindent}{2em}
\graphicspath{{fig/}}


%pdf文件设置
\hypersetup{
	pdfauthor={作者名字},
	pdftitle={文件名}
}

\title{标题}
\author{作者}
\date{\today}

\begin{document}

% 部分中文标识设置,例如 图
%%%%%%%%%%%%%%%%%%%%%%%%%%%%%%%%%%%%%%%%%%%%%%
\captionsetup[figure]{name={图},labelsep=period}
\captionsetup[table]{name={表},labelsep=period}
\renewcommand\contentsname{目录}
\renewcommand\listfigurename{插图目录}
\renewcommand\listtablename{表格目录}
\renewcommand\refname{参考文献}
\renewcommand\indexname{索引}
\renewcommand\figurename{图}
\renewcommand\tablename{表}
\renewcommand\abstractname{摘\quad 要}
\renewcommand\partname{部分}
\renewcommand\appendixname{附录}
\def\equationautorefname{式}%
\def\footnoteautorefname{脚注}%
\def\itemautorefname{项}%
\def\figureautorefname{图}%
\def\tableautorefname{表}%
\def\partautorefname{篇}%
\def\appendixautorefname{附录}%
\def\chapterautorefname{章}%
\def\sectionautorefname{节}%
\def\subsectionautorefname{小小节}%
\def\subsubsectionautorefname{subsubsection}%
\def\paragraphautorefname{段落}%
\def\subparagraphautorefname{子段落}%
\def\FancyVerbLineautorefname{行}%
\def\theoremautorefname{定理}%
\crefname{figure}{图}{图}
\crefname{equation}{式}{式}
\crefname{table}{表}{表}
%%%%%%%%%%%%%%%%%%%%%%%%%%%%%%%%%%%%%%%%%%%

\maketitle


\section{文字段落}


\lipsum[1]\footnote{这是一段随机生成的文字。}

\subsection{有序列表}

\begin{enumerate}
	\item 1
	\item 2
	\item 3
\end{enumerate}

\subsection{无序列表}

\begin{itemize}
	\item 1
	\item 2
	\item 3
\end{itemize}

% \lipsum[1]

\section{图片}

引用\cref{fig:1}.

\begin{figure}[htp]
	\centering
	\includegraphics[width=7cm]{pku.pdf}
	\caption{图片示例。}
	\label{fig:1}
\end{figure}

\section{表格}


\begin{table}[htp]
	\centering
	\caption{表格示例.}
	\begin{tabular}{cc}
		\toprule  1 & 2     \\
		\midrule
		内容1       & 内容2 \\
		\bottomrule
	\end{tabular}
	\label{tab:1}
\end{table}

\section{公式}


\begin{equation}
	\bm{F} = m\bm{a}.
	\label{eq:1}
\end{equation}
\cref{eq:1}用到了\texttt{bm}包,可以方便地加粗符号。

\begin{equation}
	\int^b_a f(x) \dif x=F(x)\bigg|^b_a.
	\label{eq:2}
\end{equation}
\cref{eq:2}用到了\texttt{commath}包,可以方便地写微分算符。

\begin{equation}
	\me ^{\pi \mi} + 1 = 0.
	\label{eq:3}
\end{equation}
\cref{eq:3}用到了自定义的\verb|\me, \mi|来表示常数$\me,\ \mi$.

\section{代码}


\begin{lstlisting}[language=python]
import numpy as np
\end{lstlisting}

\begin{lstlisting}[language=c]
#include <stdio.h>
main(){
    printf("Hello World");
}
\end{lstlisting}

\section{参考文献}

引用参考文献 .本模版设置了参考文献返回的链接,即参考文献最后的数字。\cite{olsthoorn}


% \nocite{*}

\newpage
\bibliographystyle{plain}
\clearpage
\phantomsection

\addcontentsline{toc}{section}{参考文献} %向目录中添加条目,以章的名义
\bibliography{test}

\end{document}
